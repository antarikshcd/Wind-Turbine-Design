\section{Drive train design}

\subsection{Main frame, drive train and electrical system configuration}

\subsubsection{Summary of the design}

\begin{center}
Table 41. Objectives and boundary conditions\\
\begin{tabular}{ |l|c| } 
\hline
\textbf{Parameter} & \textbf{Value/Description}  \\ 
\hline
Special considerations for the chosen system design & XX  \\ 
\hline
\end{tabular} \\
\end{center}



Figure 41. Overview of nacelle

\subsubsection{Supporting material, analyses and rationale}

\textbf{Scaling down the NREL 5MW turbine drivetrain to the AlphaWind drivetrain}

Mechanical power on the low speed shaft needed to drive the generator is augmented with the overall efficiency of the drivetrain 

\begin{equation}
    P_m = \dfrac{P}{\eta}
    \label{eq:P_m}
\end{equation}


Starting from the rated rotational speed of the low speed shaft, the torque on the low speed shaft is determined as: 

\begin{equation}
    Q_{ls} = \dfrac{P_{m}}{\omega_r}
    \label{eq:Q_ls}
\end{equation}


\newpage

\subsection{Gearbox}

\subsubsection{Summary of the design}

\begin{center}
Table 43. Objectives and boundary conditions\\
\begin{tabular}{ |l|c| } 
\hline
\textbf{Parameter} & \textbf{Value/Description}  \\ 
\hline
Speed range (LSS / HSS) & XX  \\ 
\hline
Torque (LSS / HSS) & XX \\
\hline
\end{tabular} \\
\end{center}

\begin{table}[h]
\centering
\caption{Gearbox configuration}
\label{tab:gearbox_config}
\begin{tabular}{ |l|c|c| } 
\hline
\textbf{Parameter} & \textbf{Symbol} & \textbf{Value/Description}\\ 
\hline
Gearbox type & & 3 stage multistage gearbox\\
Gearbox stage 1 & & planetary\\
Gearbox stage 2 & & parallel\\
Gearbox stage 3 & & parallel\\
Gearbox ratio & N & 75 \\
Gear ratio stage 1 & $N_1$ & 5\\
Gear ratio stage 2 & $N_2$ & 5\\
Gear ratio stage 3 & $N_3$ & 3\\
Number of teeth sun& $z_{sun1}$ & 20\\
Number of teeth planet& $z_{planet3}$ & 30\\
Number of teeth ring& $z_{ring3}$ & 80\\
Number of teeth pinion& $z_{p2}$ & 20\\
Number of teeth gear& $z_{g2}$ & 100\\
Number of teeth pinion& $z_{p3}$ & 20\\
Number of teeth gear& $z_{g3}$ & 60\\
\hline
\end{tabular} \\
\end{table}

\subsubsection{Supporting material, analyses and rationale}

The gearbox is designed as a three stages multistage gearbox with the first stage being planetary and the second and third stage being parallel stages. With this an overall gearbox ratio of 75 is achieved. 

The weight of the gearbox is estimated using the following empirical formula

\begin{equation}
    m_{gearbox} = 70.94 \cdot Q_{ls}^{0.759}
\end{equation}
where $Q_{ls}$ is the rated low speed shaft torque in kNm \cite{Fingersh2006}.



\newpage
\subsection{Generator}

\subsubsection{Summary of the design}

\begin{table}[h]
\centering
\caption{Generator configuration}
\label{tab:generator_config}
\begin{tabular}{ |l|c|c|} 
\hline
\textbf{Parameter} & Symbol & \textbf{Value/Description}  \\ 
\hline
Type of generator & & DFIG\\
Power rating & & 
Generator power factor & pf & 0.95\\
Grid frequency & $f_{grid}$ & 50 Hz  \\
Generator poles & $p$ & 6\\
Minimum generator speed & $n_{min}$ & 567 rpm\\
Synchronous speed & $n_s$ & 1000 rpm\\
Generator rated speed & $n_{rated}$ & 1077 rpm\\
Maximum generator speed & $n_{max}$ & 1250 rpm \\
Force density & $F_d$ & 30000 $\frac{N}{m}$\\
Generator volume & $V_{gen}$ & 0.31 $m^3$\\
Generator length & $L_{gen}$ & 1 m\\
Rotor radius & $R_{rot}$ & 0.31 m\\
Rotor mass & $m_{rot}$ & 2457 kg \\
Inertia generator rotor & $\Theta_{rot}$ & 122.5 $kg m^2$\\
Generator inverter rating & $x$ & 0.25\\
Savety margin against overshoot & & 16 \%\\
\hline
\end{tabular} \\
\end{table}



\begin{center}
Table 45. Objectives and boundary conditions\\
\begin{tabular}{ |l|c| } 
\hline
\textbf{Parameter} & \textbf{Value/Description}  \\ 
\hline
Power rating (think of power factor) & XX \\
\hline
Speed range (HSS) & XX \\
\hline
Torque (HSS) & XX \\
\hline
\end{tabular} \\
\end{center}

\begin{center}
Table 46. Design variables/choices\\
\begin{tabular}{ |l|c| } 
\hline
\textbf{Parameter} & \textbf{Value/Description}  \\ 
\hline
Type of generator & XX  \\ 
\hline
Number of poles of the generator & XX \\
\hline
Voltage level & XX \\
\hline
\end{tabular} \\
\end{center}



\subsubsection{Supporting material, analyses and rationale}

Generator rotor inertia
$\Theta_{rot}= \dfrac{1}{2}m_{rot}R_{rot}^2$

DFIG generators can be operated in voltage control mode (PV) and power factor control mode (PQ) \cite{Londero2012}. The DFIG generator is operated in PV mode with a power factor specified at 0.9 leading and 0.85 lagging.

\newpage
\subsection{Inverter}

\subsubsection{Summary of the design}

\begin{center}
Table 47. Objectives and boundary conditions\\
\begin{tabular}{ |l|c| } 
\hline
\textbf{Parameter} & \textbf{Value/Description}  \\ 
\hline
Voltage level & XX  \\ 
\hline
Power rating (think of power factor) & XX \\
\hline
\end{tabular} \\
\end{center}

\begin{center}
Table 48. Design variables/choices\\
\begin{tabular}{ |l|c| } 
\hline
\textbf{Parameter} & \textbf{Value/Description}  \\ 
\hline
Location (in nacelle or tower base)Type of inverter (selected from standard inverters) & XX  \\ 
\hline
Type of inverter (selected from standard inverters) & XX \\
\hline
\end{tabular} \\
\end{center}

\subsubsection{Supporting material, analyses and rationale}



\subsection{Power cables in tower}

\subsubsection{Summary of the design}

\begin{center}
Table 49. Objectives and boundary conditions\\
\begin{tabular}{ |l|c| } 
\hline
\textbf{Parameter} & \textbf{Value/Description}  \\ 
\hline
Voltage level & XX  \\ 
\hline
Power rating (think of power factor): Per phase (= per cable) & XX \\
\hline
\end{tabular} \\
\end{center}

\begin{center}
Table 410. Design variables/choices\\
\begin{tabular}{ |l|c| } 
\hline
\textbf{Parameter} & \textbf{Value/Description}  \\ 
\hline
Type of cables (selected from standard cables) & XX  \\ 
\hline
Length & XX \\
\hline
\end{tabular} \\
\end{center}
\subsubsection{Supporting material, analyses and rationale}


